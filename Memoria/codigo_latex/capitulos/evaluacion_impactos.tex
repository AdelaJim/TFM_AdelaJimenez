\chapter{ Evaluación de impactos}
\label{cap:evaluacion_impactos}

En este capítulo se procederá a evaluar los impactos del trabajo realizado, lo cual implica analizar y medir las consecuencias y efectos del proyecto propuesto. En este contexto, dicha evaluación considerará las repercusiones del proyecto en términos de su contribución social, costos, beneficios económicos y sostenibilidad. Además, se mencionarán aquellos Objetivos de Desarrollo Sostenible a los que este trabajo puede contribuir.

Con este marco de análisis, se añadirá una nueva dimensión que permitirá comprender de manera más profunda cómo este trabajo puede influir en su entorno y en las partes interesadas involucradas, proporcionando una serie de nuevos hitos para un proceso de mejora continua.

\section{Impacto social}

El desarrollo de un sistema de preparación de una estación robotizada \acrshort{NPAM} puede influir en diversas áreas de la sociedad, como se detalla a continuación:

\begin{itemize}
    \item \textbf{Generación de empleo:} A pesar de que la automatización de procesos industriales a menudo se asocia con la pérdida de empleos tradicionales, también puede crear nuevas oportunidades en áreas como la programación de sistemas inteligentes con plataformas de código abierto como ROS2. Además, las tareas de supervisión, mantenimiento de equipos y la interacción con campos como el diseño de productos o la construcción de instalaciones industriales pueden propiciar la demanda de profesionales con habilidades híbridas especializados en estas estaciones.
    
    \item \textbf{Impacto en la industria:} La integración de un sistema de fabricación aditiva de código abierto basado en ROS2 puede facilitar el acceso al mercado de manipuladores robóticos industriales para pequeñas y medianas empresas, reduciendo los costos de formación para profesionales del sector. Esto abre la puerta a la innovación empresarial y a nuevas tecnologías que optimizan estos sistemas.
    
    \item \textbf{Mejora en la seguridad:} La utilización de entornos de aprendizaje offline basados en modelos \acrshort{CAD} puede promover la implementación de sistemas de aprendizaje híbridos en estaciones de trabajo robotizadas. Este enfoque no solo facilita la integración de sistemas que incluyen al operario como parte del proceso, sino que también prioriza la seguridad como un componente fundamental de su funcionamiento.
\end{itemize}

Otros impactos sociales pueden estar relacionados con áreas más allá del ámbito industrial. Por ejemplo, el desarrollo de una plataforma basada en ROS2 para la fabricación robotizada puede plantear desafíos legales relacionados con la propiedad intelectual y las regulaciones de seguridad laboral.


\section{Impacto económico}
El desarrollo de una estación \acrshort{NPAM} robotizada puede traer varios económicos positivos como:
\begin{itemize}
    \item \textbf{Reducción de costes de producción:} Los sistemas \acrshort{NPAM} conllevan una disminución del material empleado en el proceso gracias a su capacidad para aprovechar la geometría de la plataforma de impresión a su favor, optimizando la geometría y propiedades mecánicas del producto final. Es decir se pueden desarrollar componentes complejos de manera eficiente y reduciendo el material empleado. Por otro lado, la integración de sistemas automatizados puedo optimizar la mano de obra reduciendo los costes de operarios asociados.
    \item \textbf{Innovación y desarrollo de nuevos productos:} La capacidad de las estaciones \acrshort{AM} robotizadas para imprimir geometrías complejas y personalizas abre nuevas oportunidades en el diseño de productos. Es decir, se pueden crear nuevas gamas de productos adaptados específicamente a las necesidades del mercado y del cliente. Este enfoque resulta favorable para la creación de nuevas fuentes de ingresos y aumentar la competitividad de las empresas del sector.
    \item \textbf{Impulso a la economía local y regional:} La adaptación de tecnologías \acrshort{NPAM} más accesibles a usuarios generalistas puede promover el crecimiento de sectores de la industria local y regional. Este crecimiento ve como principal aliciente la demanda de servicios de impresión avanzada, así como también las labores de mantenimiento y operación especializada en equipos robotizados. Como punto adicional, se puede reducir la dependencia de proveedores externos de componentes y productos manufacturados, lo que favorecería la fortaleza económica de la región.
\end{itemize}

\section{Impacto medioambiental}
La robotización de procesos de fabricación supone una serie de impactos positivos en el cuidado del medio ambiente como son:
\begin{itemize}
    \item \textbf{Reducción de residuos:} El uso de un sistema robotizado de 6 \acrshort{DOF} favorece la optimización del material depositado, reduciendo el uso de soportes para elementos voladizos o de material de purga en la etapa de preparación. 
    \item \textbf{Ahorro de energía:} La fabricación robotizada optimiza los movimientos del manipulador robótico de forma que se emplee la cantidad mínima de energía para realizar el trazado planificado y/o la herramienta recorra la mínima distancia. En ambos casos, el consumo energético disminuye frente a una situación que no emplea estos sistemas y se fomenta un consumo eficiente de los recursos energéticos.
    \item \textbf{Reducción de emisiones de gases de efecto invernadero:} La reducción de material depositado y el ahorro energético asociado al proceso, disminuye la demanda de material y energía. Es decir, se minimiza la energía empleada en los procesos de producción, transporte y distribución de las materias primas para la estación, así como las emisiones de CO$_2$ asociadas
    \item \textbf{Diseño eficiente:} La flexibilidad de los procesos \acrshort{NPAM} y la optimización mecánica asociada a ellos supone una reducción del material empleado y del número de operaciones robotizadas, disminuyendo el consumo de materias primas y recursos energéticos.
\end{itemize}

\section{Contribución a los Objetivos de Desarrollo Sostenible}
Los Objetivos de Desarrollo Sostenible (ODS) son un conjunto de 17 metas globales adoptadas por todos los Estados Miembros de las Naciones Unidas en 2015, como parte de la Agenda 2030 para el Desarrollo Sostenible \cite{web_ods}. Estos objetivos buscan abordar los principales desafíos mundiales, incluyendo la pobreza, la desigualdad, el cambio climático, la degradación ambiental, la paz y la justicia. Los ODS proporcionan una hoja de ruta para alcanzar un futuro sostenible y equitativo para todas las personas, equilibrando las dimensiones económica, social y ambiental del desarrollo humano.

La introducción de un modelo de estación robotizada (junto a su respectivo flujo de trabajo) para procesos de fabricación \acrshort{NPAM} podría contribuir a los siguientes ODS:

\begin{itemize}
    \item \textbf{4 - Educación de calidad:} La estación facilita el acceso a tecnologías avanzadas de fabricación, lo que puede ser utilizado en entornos educativos para formar a estudiantes en nuevas tecnologías y habilidades técnicas.
    \item \textbf{8 - Trabajo decente y crecimiento económico:} Se facilita el acceso a tecnologías de fabricación avanzadas, impulsando la innovación y creando oportunidades para trabajos más cualificados y sostenibles.
    \item \textbf{9 - Industria, innovación e infraestructura:} Se promueve la innovación en la fabricación aditiva y la integración de tecnologías avanzadas en la industria, mejorando la infraestructura tecnológica.
    \item \textbf{10 - Reducción de desigualdades:} Ofrece acceso a tecnologías de fabricación avanzadas a una diversidad de usuarios, reduciendo las brechas tecnológicas y sociales.
    \item \textbf{11 - Ciudades y comunidades sostenibles:} La fabricación aditiva y la reducción de residuos y consumo de energía apoyan la creación de ciudades más sostenibles, con prácticas de producción más responsables y eficientes.
    \item \textbf{13 - Acción por el clima:} La implementación de sistemas robotizados contribuye a la mitigación del cambio climático. Este fenómeno se da mediante la reducción de emisiones, desechos y consumos energéticos.
    \item \textbf{17 - Alianzas para lograr los objetivos:} Se fomenta un ambiente colaborativo que puede facilitar la cooperación entre diferentes sectores y comunidades, promoviendo alianzas estratégicas para el desarrollo sostenible.
\end{itemize}



